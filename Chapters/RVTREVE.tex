\section{Improving Completeness Using Tools Integration}
The first contribution of my research is to integrate RVT and REVE and by doing so produce a technique that is more complete than each one of them on its own. More specifically, the new algorithm will be able to prove equivalence for at least each case RVT or REVE could. The feature that let us to combine both engines is the doom mechanic we have described is section \ref{sec:rvtreview}. One of its advantages is that it let us integrate other tools into the decomposition process. We would like to use REVE in three cases. The first is when RVT fails, REVE will be invoked and will try to prove equivalence for the same pair. The second is when we encounter a doomed function in the call graph that has a mapping to a function in the second program. That means that one of its descendants was a recursive function and could not be proven equivalent to any other function in the second program. As we said, this nullifies RVT capability of proving equivalence. As stated in section \ref{sec:relatedwork}, REVE works differently and isn't affected by this limit. Hence, if we execute REVE on the pair, and it will succeed, we will note both functions and their ancestors as not doomed. From this pair and upward in the call graph, RVT can be used again. The third case is when a node is doomed due to lack of mapping. The problem here lies in the mapping algorithm used to generate the input for RVT decomposition algorithm. Once the algorithm fails to map a recursive function it stops trying to map its ancestors as well. This made sense when the only proof engine was RVT, but now it has become an obstacle we must overcome. A new mapping algorithm needs to be chosen, one that can map ancestors even in the case their descendants could not be mapped. A trivial solution could be mapping all the functions with the same signature. I will experiment whether this idea can hold in real life programs or perhaps a more sophisticated idea must be conceived. Once we have figured a way to expand our mapping, REVE will be able to check for their equivalence. And as in the second case, if it succeeds, we will be able to execute RVT from hereon. 

The combined algorithm is shown in Algorithm \ref{alg:Prove}. It receives the mapping as input, and therefore is agnostic to our mapping generation algorithm. There are several key changes it is important to note. In line \ref{step:whileOneSide} we iterate over MSCC nodes in one graph rather than mapping pairs as in Algorithm \ref{alg:OldProve}. That is because the original algorithm assumed the mapping must be \emph{connected} - every mapped pair is either a map between leaves in the call graphs or the children of both paired functions are mapped as well. We want to allow now an \emph{unconnected} mapping – pair of functions that their descendants are not necessarily mapped. Iterating over such a mapping is achieved by traversing only one call graph. Because the mapping relation is symmetric, and each pair contains a node from this call graph, it is assured we will cover all the pairs in the mapping. Another significant change is the addition of "RVT enable" which helps to track for each MSCC whether RVT can be invoked to check its equivalence to its mapped MSCC. This flag is updated throughout Algorithm \ref{alg:Prove} repetitively in the cases that were described earlier in this section.  

\noindent
\begin{algorithm}
\begin{minipage}{\linewidth}
\begin{algorithmic}[1]
\Function{Prove}{Programs $P,P'$, map between functions $map_f$}
\State \label{step:inline} Inline nonrecursive nonmapped functions;
\State \label{step:generate} Generate MSCC DAGs $MD_1$, $MD_2$ from the call graphs of $P, P′$; 
\State \label{step:markEnabled} mark all the nodes in $MD_1$ as "RVT enabled";
\State If possible, generate a bijective map $map_m$ between nontrivial nodes in $MD_1$ and $MD_2$ that is consistent with \mbox{~~~~~} $map_f$ (it is desirable but not necessary to add pairs of trivial nodes to $map_m$). Otherwise Abort. 
\While {$\exists m1 \in MD_1$ that is uncovered, and its children are “Covered”} \label{step:whileOneSide}
\If {$\exists m2 \in MD_2\ s.t.\ \exists \pair{m1,m2} \in map_m $} \label{step:findMapping}
   \If {$m,m'$ are trivial} \label{step:m1}
     \State Let $f,f'$ be the functions in $m,m'$, respectively;
     \If {\alg{$check^r$}$(m1,f,f',\{\pair{f,f'}\})$} \label{step:Check}
      \alg{markEquivalent}$(f1, f2, m1)$; \label{step:markTrivial}
    \EndIf
   \Else
    \label{step:select} 
    \State Select non-deterministically $S \subseteq \{\pair{f,f'}\mid \pair{f,f'}\in map_f(m)\}$ \newline \mbox{~~~~~~~~~~~~~~~~~~~~~~~} that intersect every cycle in $m1$ and $m2$;
      \For {each $ \pair{f,f'} \in S$}\label{step:forall}
        \If  {$\lnot \alg{check^r}(m1,f,f',S)$} \label{step:abort}
         \State Mark m1 and all its ancestors as not "RVT enabled"; \label{step:doom}
        \State Break;
        \EndIf
      \EndFor
      \If {m1 is marked "RVT enabled"} \label{step:markNonTrivial}
      \For {each $ \pair{f,f'} \in S$}\ {\alg{MarkEquivalent}$(f1, f2, m1)$}
      \EndFor
      \EndIf
      \EndIf
      \Else
      \If {m1 is not trivial }
      \State Mark m1 and all its ancestors as not "RVT enabled";
      \EndIf
   \EndIf
   \State Mark m1 as “Covered”;
\EndWhile
\EndFunction
\end{algorithmic}
\end{minipage}
\caption{A decomposition algorithm integrating RVT and REVE}
\label{alg:Prove}
\end{algorithm}
\begin{algorithm}
\begin{minipage}{\linewidth}
\begin{algorithmic}[1]
\Function {$Check^r$}{$MSCC\  m1,function\ f, function\ f′, set\ of\ pairs\ S$}
\label{step:Checkr2}
\If{m1 is marked as "RVT enabled" }
\If{f and f′ are syntactically equivalent and all their children are either marked “Equivalent” or in S }
\State \Return true;
\EndIf
\If{CBMC ($Check-block^r$(f,f′,S))}
\State \Return true;
\EndIf
\EndIf
\State Let $S_{equal}$ be the set of all the functions proved equal so far in our execution
\State \Return LLREVE$(f,f',S_equal)$
\EndFunction
\end{algorithmic}
\end{minipage}
\caption{A function called by \alg{Prove} for checking the equivalence of two input functions that are part of MSCCs. This version execute both proving engine - RVT and REVE. $check-block^r$ is the same C program as in the old \alg{$check^r$}.}
\end{algorithm}
\begin{algorithm}
\begin{minipage}{\linewidth}
\begin{algorithmic}[1]
\Function {MarkEquivalence}{function f, function f′, MSCC m}
\label{step:markEquivalent}
\State Mark f1 and f2 as "Equivalent";
\State Mark m1 as "RVT enabled"; 
\For{each ancestor g of m1 in }
\If{all the sons of g are marked as "RVT enabled" }
\State Mark g as "RVT enabled";
\EndIf
\EndFor
\EndFunction
\end{algorithmic}
\end{minipage}
\caption{A function called from \alg{Prove} to mark the equivalence of $m1$ and its mapping in $MD_2$.}
\end{algorithm}

\begin{figure}
\label{fig:callgraphs}
\begin{center}
\begin{minipage}{.2\textwidth}
\begin{tikzpicture}[>=stealth',shorten >=1pt,node distance=1.5cm,on grid,semithick ]
\tikzstyle{mapped0} = [circle,draw=black,fill={rgb:black,2;white,2}]
\tikzstyle{mapped1} = [circle,draw=black,fill={rgb:black,1;white,2}]
\tikzstyle{mapped2} = [circle,draw=black,fill={rgb:black,1;white,4}]
\tikzstyle{mapped3} = [circle,draw=black,fill={rgb:black,1;white,8}]
    \node[mapped0] (main) {$main$};
    \node[mapped1] (f1) [below =of main] {$f1$};
    \node[state] (f2) [below left =of f1] {$f2$};
    \node[state] (f4) [below right =of f1] {$f4$};
    \node[mapped2] (f3) [below left =of f2] {$f3$};
    \node[mapped3] (f5) [below left =of f4] {$f5$};
    \node[state] (f6) [below right =of f4] {$f6$};
    
    \tikzset{mystyle/.style={->}}
    \tikzset{every node/.style={fill=white}}
    \draw (main) edge [mystyle]  (f1);
    \draw (f1) edge [mystyle]  (f2);
    \draw (f1) edge [mystyle]  (f4);
    \draw (f2) edge [loop left]  (f2);
    \draw (f2) edge [mystyle]  (f3);
    \draw (f4) edge [mystyle]  (f5);
    \draw (f4) edge [mystyle]  (f6);
    \draw (f6) edge [loop right]  (f6);
\end{tikzpicture}
\end{minipage}
\hspace{3.5cm}
\begin{minipage}{.2\textwidth}
\begin{tikzpicture}[>=stealth',shorten >=1pt,node distance=1.5cm,on grid,semithick ]
\tikzstyle{mapped0} = [circle,draw=black,fill={rgb:black,2;white,2}]
\tikzstyle{mapped1} = [circle,draw=black,fill={rgb:black,1;white,2}]
\tikzstyle{mapped2} = [circle,draw=black,fill={rgb:black,1;white,4}]
\tikzstyle{mapped3} = [circle,draw=black,fill={rgb:black,1;white,8}]
    \node[mapped0] (main') {$main'$};
    \node[mapped1] (f'1) [below =of main'] {$f'1$};
    \node[state] (f'2) [below left =of f'1] {$f'2$};
    \node[state] (f'4) [below right =of f'1] {$f'4$};
    \node[mapped2] (f'3) [below left =of f'2] {$f'3$};
    \node[mapped3] (f'5) [below left =of f'4] {$f'5$};
    
    \tikzset{mystyle/.style={->}}
    \tikzset{every node/.style={fill=white}}
    \draw (main') edge [mystyle]  (f'1);
    \draw (f'1) edge [mystyle]  (f'2);
    \draw (f'1) edge [mystyle]  (f'4);
    \draw (f'2) edge [loop left]  (f'2);
    \draw (f'2) edge [mystyle]  (f'3);
    \draw (f'4) edge [mystyle]  (f'5);
    \draw (f'4) edge [loop right]  (f'4);
  
\end{tikzpicture}
\end{minipage}
\end{center}
\caption{$MD_1$ and $MD_2$. Mapped nodes are colored with their unique grey. }
\end{figure}

We will now show an example where \alg{Prove} (Algorithm \ref{alg:Prove}) achieves better results in term of completeness than \alg{OldProve} (\ref{alg:OldProve}). 
\begin{example}
Consider $MD_1$ and $MD_2$ in Figure \ref{fig:callgraphs}. $map_m$ is defined as $map_m = \{ \pair{main,main'},\pair{f1,f'1}, \pair{f3,f'3},\pair{f5,f'5}\}$. Note that $map_m$ is an \emph{unconnected} mapping ($\pair{f1,f'1}$ have no mapped sons) and could not be generated by RVT before applying the changes we have proposed here. \alg{OldProve} from Algorithm \ref{alg:OldProve} would have faced two problems when checking equivalence for this example. First, it would have stopped when it has finished checking equivalence for $\pair{f3,f'3}$ and $\pair{f5,f'5}$. After covering those pairs, \alg{OldProve} cannot find any pair that is uncovered and its children are "Covered" and will terminate its execution. Our new \alg{Prove} from Algorithm \ref{alg:Prove} resolves this issue by traversing only $MD_1$. Every node in $MD_1$ will be visited and will be checked for equivalence against its pair in $MD_2$ if exists. The second problem exists even after changing the traversing algorithm of \alg{OldProve}. Once it has reached $f6$ it would have aborted. That is because RVT cannot handle an equivalence checking of functions where one of its children is recursive and is not equivalent to any other function in the other program. In \alg{Prove}, encountering $f6$ would doom all its ancestors, but it will not abort as LLREVE will be used to check their equivalence. In our example, $f4$ will be encountered but it has no mapping so no check will be performed here. Next, $f1$ will be visited, and because it is mapped to $f'1$, LLREVE will be executed to check their equivalence. For the sake of this example let us assume it has succeeded. Now that all the recursive children of main are equivalent (i.e. $f1$), $main$ will be undoomed and RVT can be used to check the equivalence of $\pair{main,main'}$. Those two virtues of \alg{Prove} over \alg{OldProve} are a major step forward towards a more complete generator for proofs of equivalence .
\end{example}